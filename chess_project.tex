\documentclass{article}

% if you need to pass options to natbib, use, e.g.:
%     \PassOptionsToPackage{numbers, compress}{natbib}
% before loading neurips_2019

% ready for submission
% \usepackage{neurips_2019}

% to compile a preprint version, e.g., for submission to arXiv, add add the
% [preprint] option:
%     \usepackage[preprint]{neurips_2019}

% to compile a camera-ready version, add the [final] option, e.g.:
     \usepackage[final]{neurips_2019}

% to avoid loading the natbib package, add option nonatbib:
%     \usepackage[nonatbib]{neurips_2019}

\usepackage[utf8]{inputenc} % allow utf-8 input
\usepackage[T1]{fontenc}    % use 8-bit T1 fonts
\usepackage{hyperref}       % hyperlinks
\usepackage{url}            % simple URL typesetting
\usepackage{booktabs}       % professional-quality tables
\usepackage{amsfonts}       % blackboard math symbols
\usepackage{nicefrac}       % compact symbols for 1/2, etc.
\usepackage{microtype}      % microtypography
\usepackage{graphicx}

\title{Retrieving Similar Chess Positions}

\author{%
  Jacob Leppek \\
  MSCAPP\\
  \texttt{leppekj@uchicago.edu} \\
\And
Ashu Tayal \\
MPP\\
\texttt{ashutayal@uchicago.edu} \\
}

\begin{document}

\maketitle

\begin{abstract}
   Improvement in chess relies on pattern recognition; as such, the ability to search
  games by position is key to understanding different responses and strategies. Yet the
  number of legal chess positions is estimated to be possibly $10^{40}$. As such, very few
  of these positions have actually been played. To accommodate for this, we explore a number
  of approaches to approximating positions that are highly similar to the initial query. We
  encode positions as vectors; use k-means to group similar positions together to limit the
  number of positions searched; and finally return positions with the highest cosine similarity
  scores from the searched cluster.
\end{abstract}

\section{Introduction}

    It is often beneficial for a chess player during a game to know whether previously archived chess games lead to positions approximately similar to the current position. It allows players to improve through repetition and understand how positional advantages emerge with seemingly inconsequential differences in piece position. It can, thus, be really helpful in deciding the next set of moves. Kotov, the Soviet chess grand master, opines that if one can recall similar positions from earlier games, then it is “easier to reach an assessment of how things stand, and to hit upon the correct plan oranalyze variations”. Many chess databases exist online that let users retrieve games based on position. This allows players to input a position and see what games that position emerged from as well as what moves were played before and after. But no free solution allows exists for letting users search for games with similar – but not exact – positions.

 Our goal is to understand how linear algebra could provide a possible solution to returning similar positions from a given position, allowing users to generate new games with subtle differences that they can use to improve their positional understanding. Given the large possible set of chess positions, this also allows users to find the most similar board positions when no exact match is possible. We are curious about how linear algebra can be used to develop similarity scores for board positions and recommend new positions to players for study. We would like to understand how positions can be efficiently represented as matrices, if there are ways to simplify the representation of sparse positions (2-3 pieces among 64 squares), and how effective this approach is when dealing with large data sets.We used a subset of games obtained from lichess.org in PGN (text) format. This labeled data set consists of complete games with metadata on players, scores, times, and computer evaluations.

 A 2014 paper from Dublin City University provides a discussion of approximate searching methods from an information retrieval perspective.\footnote{http://doras.dcu.ie/20378/1/ganguly-sigir2014.pdf} The authors briefly discuss encoding the chess board and pieces as feature vectors before dismissing it as inconvenient relative to textual representations for search purposes. They discuss what features should be used when developing a similarity score, and construct functions to calculate and combine these features for database retrieval purposes. We have not come across other academic papers that explicitly discuss this topic.

\section{Background and Methodology}

\subsection{Chess algebraic notation}
\subsection{PGN Notation}
\subsection{FEN notation}
\subsection{K-means}
\subsection{Cosine similarity scores}
% PGN Notation
% FEN notation
% K-means
% Cosine similarity scores


\section{Headings: first level}
\label{headings}

All headings should be lower case (except for first word and proper nouns),
flush left, and bold.

First-level headings should be in 12-point type.

\subsection{Headings: second level}

Second-level headings should be in 10-point type.

\subsubsection{Headings: third level}

Third-level headings should be in 10-point type.

\paragraph{Paragraphs}

There is also a \verb+\paragraph+ command available, which sets the heading in
bold, flush left, and inline with the text, with the heading followed by 1\,em
of space.

\subsection{Figures}

\begin{figure}
  \centering
  \includegraphics[width=0.4\columnwidth]{elbowcurve.png}
  \caption{Elbow curve for 8 different cluster sizes.}
\end{figure}

All artwork must be neat, clean, and legible. Lines should be dark enough for
purposes of reproduction. The figure number and caption always appear after the
figure. Place one line space before the figure caption and one line space after
the figure. The figure caption should be lower case (except for first word and
proper nouns); figures are numbered consecutively.

You may use color figures.  However, it is best for the figure captions and the
paper body to be legible if the paper is printed in either black/white or in
color.


\section{Final instructions}

Do not change any aspects of the formatting parameters in the style files.  In
particular, do not modify the width or length of the rectangle the text should
fit into, and do not change font sizes (except perhaps in the
\textbf{References} section; see below). Please note that pages should be
numbered.

\section*{References}

\medskip

\small

[1] Alexander, J.A.\ \& Mozer, M.C.\ (1995) Template-based algorithms for
connectionist rule extraction. In G.\ Tesauro, D.S.\ Touretzky and T.K.\ Leen
(eds.), {\it Advances in Neural Information Processing Systems 7},
pp.\ 609--616. Cambridge, MA: MIT Press.

[2] Bower, J.M.\ \& Beeman, D.\ (1995) {\it The Book of GENESIS: Exploring
  Realistic Neural Models with the GEneral NEural SImulation System.}  New York:
TELOS/Springer--Verlag.

[3] Hasselmo, M.E., Schnell, E.\ \& Barkai, E.\ (1995) Dynamics of learning and
recall at excitatory recurrent synapses and cholinergic modulation in rat
hippocampal region CA3. {\it Journal of Neuroscience} {\bf 15}(7):5249-5262.





\end{document}
